\section{Method}

This experiment uses a two-dimensional, aquatic environment to test
what kinds of morphological changes may accelerate evolution. The
morphological forms---inspired by frog metamorphosis---have been
selected such that the conservation of the infant form to the adult
form may be varied. In one test case, the robot begins as a
``tadpole'' bearing only a tail. It transforms into a ``frog'' with
four limbs. Its task is to swim to a given
target. Figure~\ref{morphology}

%\begin{wrapfigure}{r}{0.5\textwidth}
\begin{figure}[h]
  \centering
  \figtitle{Body Plans} 
  \includegraphics[scale=0.5]{fig/forms.pdf} 
  \vspace{-30pt}
  \caption[Body plans]{\label{morphology}Body plans parameterised by
    tail length $l_t$ and foot length $l_f$. a) represents the infant
    ``tadpole'' form, and b) represents the adult ``frog'' form. }
\end{figure}
%\end{wrapfigure}

\subsection{Physics Model}

This experiment uses the following physical model to simulate a
``frog'' in an two-dimensional aqautic environment.  The aim of the
simulation is to provide a water-like environment, but it is not
intended to provide a realistic environment such that a controller
evolved in simulation could be transferred to a real
robot.  \begin{comment} However, the same technique could be applied
  to a real robot.\end{comment}

The body is composed of six rigid bodies: one central body, one tail
segment, four feet segments.  The tail and feet are connected to the
central body by pinwheel joints.  Eight configuration variables
$\{q_1, q_2, \ldots, q_8\}$ describe the body as shown in
Figure~\ref{confvars}.  The position of the body is denoted by the
vector $(q_1, q_2)$.  The angle of the central body measured counter-clockwise
to the $\bf \hat j$ axis denoted by $q_3$.  The angle of the tail and
four feet are denoted by $q_4, \ldots, q_8$, respectively.  Eight
corresponding motion variables $\{u_1, u_2, \ldots, u_8\}$ describe
the generalised speeds of the body where $u_i = \frac{d q_i}{d t}$.

\begin{figure}  
  \centering
  \figtitle{Diagram of Configuration Variables} 
  \includegraphics[scale=0.6]{fig/confvars.pdf} 
  \caption[Diagram of configuration variables]{\label{confvars}Diagram
    of the configuration variables $\{q_1, q_2, \ldots, q_8\}$ that
    fully describe the physical state of the individual at a given
    time}
\end{figure}

\subsection{Forces}

For each limb a drag force $\bv F_D$ opposes its direction of motion
which are given by Equation~\ref{drag-force-eq} where $\rho$ is the
density of the fluid, $c_d$ is the drag coefficient shown in
Figure~\ref{drag-force}.  The width of the limb $w$ is set to zero
since the drag contributed by the limb when the velocity is
perpendicular to the normal is not considered to be significant.

\begin{eqnarray}
  A &=& l\,d~ |\bhv v \cdot \bhv n| + l\, w~|\bhv v \times \bhv n|\\
  w &=& 0 \\
  \bv F_D &=& -\frac{1}{2} \rho\,c_d~||\bv v||^2\,A~\bhv v \\
  \bv F_D &=& -\frac{1}{2} \rho\, c_d~ l\,d~|\bv v \cdot \bhv n|\bv v \label{drag-force-eq} 
\end{eqnarray}


\begin{figure}[h]  
  \centering
  \figtitle{Diagram of Drag Force} 
  \includegraphics[scale=0.7]{fig/drag-force.pdf} 
  \vspace{-45pt}
  \caption[Diagram of drag force]{\label{drag-force}For each limb its
    length $l$, width $w$, depth $d$ (not shown), velocity $\bv v $,
    normal vector $\bhv n$ determine its drag force $\bv F_D$.}
\end{figure}

In addition, a drag force and drag torque is exerted on the central
body.  The full equations of motion are given in the
Appendix~\ref{physeqs-sec}.

\subsubsection{Collisions}

Interbody collisions are permitted, so limbs may freely move through
one another.\footnote{This is not thought objectionable because one
  can imagine constructing a robot with limbs arranged on planes such
  that they could pass each other unobstructed.}  However, the limbs
are constrained to not penetrate the central body.  When the angle of
a limb reaches $|q| = \frac{\pi}{2}$, an impulse force causes the
velocity to change directions and slow down.


\subsection{Controller}

The controller used for the robot is a Continuous Time Recurrent
Neural Network (CTRNN).  The dynamics of a neuron $y_i$ is given by
Equation~\ref{ctrnn-eq} with time constant $\tau_i \in [0.1, 100]$,
weights $w_{ji} \in [-4, 4]$, bias $\theta_i \in [-2, 2]$, sensors
$s_j \in \R $, and sensor weights $n_{ji} \in [-4, 4]$.

\begin{eqnarray}
  \tau_i \frac{d y_i}{dt} &=& -y_i + \sum_{j = 1}^m w_{ji} \sigma(y_j - \theta_i) + \sum_{j=1}^s n_{ji} s_j \text{ for } i \in [1,5] \label{ctrnn-eq} \\
  \sigma(x) &=& \frac{1}{1 + e^{-x}}
\end{eqnarray}

The following sensors are used: $\{s_1, s_2, \ldots, s_{14}\} =
\{||(u_1, u_2)||, u_3, ||target - (q_1, q_2)||, \text{angle to
  target}, q_4, u_4, q_5, u_5, q_6, u_6, q_7, u_7, q_8, u_8\}$.  The
$s_1$ and $s_2$ sensors provide the translational and rotational speed
of the body.  A range finder for a target is given by the $s_3$ and
$s_4$ sensors.  Proprioceptive sesnsors are given by the $s_5, s_6,
\ldots, s_{14}$ sensors.

Five motor neurons are used ($m = 5$).  Each neuron exerts a torque on
its associated limb.  The torque for each limb $T(q_i)$ is given in
Equation~\ref{limb-torque}.

\begin{eqnarray}
  T(q_{i + 3}) &=& T_{max}~clip(y_i) \text{ for } i \in [1,5] \label{limb-torque} \\
  clip(x) &=& \begin{cases}
              1 & x > 1 \\
              -1 & x < -1 \\
              x & \text{otherwise} 
              \end{cases}
\end{eqnarray}

\subsection{Genetic Encoding}

The CTRNN controller is specified by a real vector gene $\bv g \in [0,
  1]^{105}$.  Each gene component $g_k$ is associated with one and
only one of the CTRNN parameters $\tau_i, w_{ij}, \theta_j, \text{ and }
n_{ij}$.  The $w_{ij}, \theta_j, n_{ij}$ parameters are linearly
mapped from the domain of the gene $[0,1]$ to the domain of each
parameter.  The $\tau$ parameter uses a non-linear mapping $\tau_i =
10^{-1 + 3 g_k}$.  

\subsection{Morphological Change}

Morphological change is considered over phylogenetic and ontological
time.  Two adult forms are evolved: A) frog with a tail $(\red{l_t},
\blue{l_f}) = (1,1).$ B) frog without a tail $(\red{l_t}, \blue{l_f}) =
(0,1).$ The control case is no morphological change denoted $An$ and
$Bn$.  The first experimental cases concern phylogenetic change
denoted $Ap$ and $Bp$, which are divided into phases $p_i$.  The
second experimental case concerns ontological change denoted $Ao$ and
$Bo$.  Figure~\ref{morph-regiment} shows the tail length $\red{l_t}$ in
red and feet length $\blue{l_f}$ in blue for each experimental case.
The exact values used are recorded in the
Appendix~\ref{morph-regiment-values}.

\begin{figure}
  \centering
  \figtitle{Variations of Morphological Change} 
  \includegraphics[scale=1.0]{fig/morph-regiment.pdf} 
  \vspace{-15pt}
  \caption[Variations of morphological
    change]{\label{morph-regiment}An, Bn have no morphological change;
    Ap, Bp have phylogenetic change; Ao, Bo have ontological change.}

\end{figure}

\subsection{Controller Variation}

In the test cases $Bp$ and $Bo$ the infant form is not conserved in
the adult form.  However, the controller may conserve some behaviour
acquired in the infant form that is useful in the adult form, which
may also accelerate evolution.  To determine whether this may happen,
two types of CTRNN controllers are considered: 1) the ``lobotomised''
controller, which has two independent CTRNNs, one for the tail; one
for the feet. 2) the ``non-lobotomised'' controller, which has one
CTRNN that controls both the tail and feet.  These are shown in
Figure~\ref{ctrnn-figures}.

\begin{figure}[h]
  \centering
  \figtitle{Variation of CTRNN Controllers} 
  \includegraphics[width=5in]{fig/ctrnn-figures.pdf}
  \vspace{-15pt}
  \caption[Variation of CTRNN controllers]{\label{ctrnn-figures}a)
    Fully connected CTRNN controller. b) Independent CTRNN controllers
    for the tail and legs.}
\end{figure}

\subsection{Tasks}

Tasks are over HERE!


